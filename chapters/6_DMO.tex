\setchapterpreamble[u]{\margintoc}
\glsresetall % reset glossary
\todo{tables always small}
\chapter{Optimizing the layout of the modules in space}
Introduction
\section{Optimizing the modules' layout using a clustering algorithm}

\subsection{Optimization formulation}

\subsection{A clustering algorithm to identify similarly behaving subdomains}

\section{Optimize the modules' layout using a modified DMO algorithm}

\todo{difference with tugilimana, we take into account the buckling whensolving the first subproblem of module layout, we can have an empty subdomain andwe use a gradeint descent algo}

\subsection{Variables penalization schemes}

\subsection{Optimization formulation}

\paragraph{Sensitivity analysis}

\subsection{Optimization initialization}

\section{Numerical application}

\subsection{Clustering algorithm}
symmetric domain without buckling to validate the clustering ability to identify mechanical behaviour. optimize with variable linking

\subsection{On the importance of the empty module}

\paragraph{Penalty schemes validation}

\paragraph{Layout of fixed cells and empty}
same as manual clustering variable linking, but with empty cells to show the importance of the second algo

\subsection{Parametric study on the number of the modules and the quantity of subdomains}
on the tugilimana load case. 

\paragraph{Tugilimana validation - Number of modules}
\todo{plot of the volume agains the number of module topologies. metti in evidenza the two extremes, the monolitic and the fully modular}

\paragraph{The quantity of subdomains}

\subsection{On the importance of the local buckling}
Tugilimana vs the same but with local buckling

\paragraph{On the different scales of local buckling}
as the cells becomes smaller, the failure type change as well

\subsection{3D}
Simply supported 3D beam

\section{Conclusion}