\chapter*{Conclusion and perspectives}
\phantomsection
\addcontentsline{toc}{chapter}{Conclusion and perspectives}
\markboth{Conclusion and perspectives}{Conclusion and perspectives}%

\section*{Conclusion}
In the aerospace sector the reduction of weight plays a fundamental role, due to the high interdependency between weight and lift. In aviation, reducing aircraft weight is crucial because it lowers the wing loading, improving aerodynamic efficiency, maneuverability, and fuel efficiency. This is because not only is an economic concerm but more and more an environmental defis. to answer to tese defis, new concepts such the flying wing or the use of transonic truss braced wings with high elongation have been proposed. a shared need of these concepts is to use thin dry wings with high aspect ratio. A possible design concept that could fitt well in this context is to use lattice structures as primary structure of these wings, thanks to their inherent low mass and all the benefit of a modular costruction. The problem with this concept is that there is a lack of standardized method to design and optimize them. For all of these reason, this thesis presented the developement of a design method and an optimization algorithm for ultralight lattice structures. 

In \chpref{} we conduted a throughtful comparison between two of the most pertinent optimization framework for optimizing ultrlight structures: the density based topology optimization and the \gls{tto} approaches. First, a shared volume minimization with material resistance constraints formulation is developed for both methods. Then, after having explored the modelizationdifference of the two approaches--expecially the difference between nand and sand approaches, a series of numerical test have been performed using the two dimensional l shaped beam as test case. To obtain different values of volume fraction of the optimized structure, we modified gradually the material properties--\ie a high value of strenght is associated witha low volume fraction--. We tracked for every test the volume fraction, the compliance, the material resistance and the comutational time, and we discovered principally two limits in the range of applicability of the two methods: the density based topology optimization approaches inevitably reaches a limit at low volume fractions as the continuous discretization needs a continuous refinelment to go to low fractions. This greatly adds to the comutational cost of the optimization, as not only the \gls{fea} takes longer, but also because the number cof design variables and sensityvity analysis is concerded. On the other hands, the \gls{tto} deos not show this problem, but it have a diametral problem: if we go towards very high volume fractions the radius of the cross sectional areas is too big and the truss idealization lose sense. Concerning the comutational time, we have observed that the tto approach, thanks to being linear, permits a. a last important difference is that with a continuous discretization more elements are needed if we want to have more detials in the opptimized structure, while if we are using tto with a ground structure discretization, this is true no more. Thanks to all these observations, we decided to carry on the work on TTO as it is fitting really well our idea to work on ultralight structures.

Now that the choiche to use tto is done, in chapter 3 we want to address the main limitations of this method and to improve the applicability. First, we concieved a constraint on the minimu slenderness of the active bars of the optimized structures, tring to stretch the domain of applicability of the method. Later, we interessed on the addition of local buckling constraints to the problem, indispensabile for very lighweight structures. special attention was needed in this case as we wanted to solve the broblem of nodal stability of copressed bars--referred as buckling chains or topological buckling in the literature--. Finally, we modeled kinematic compatibility constraints, needed to treat more complex problems that results in a statically inadmissible structure-- such in the case of using multiple load cases or imposed symmeties--. We formulated then the full tto formulation with all these addition, taht could be solved also for multiple load cases. We observed, however that naively implementing this formulation on an NLP solver  was unsuccesfull-- \ie the solution was presentin always a lot of candidates, the optimizer was uncapable of discerning which bar should be active in the solution--, as the problem is extremely multimodal and it was not capable of dealing with all of these contraints contemporaneamente. For that reason we developed an innovtive two step optimimization algorithm, utilizing a relaxed problem to generate an initial approximate solution for subsequent optimization using a complete formulation. The Reinitialization heuristic is also proposed to reduce the influence of the starting point on optimization results. thanks to that, we were able to show how the proposed algorithm is capable of efficacely dealing with some problems of the literature, as the ten-bar truss, where we showed zero influence of the starting point with 100 random initializations, or to find better solution to the 2D cantilever beam problem proposed by Achzingher. Finally we tested the proposed algorithm on more complex load cases, suc as a two dimensional truss with multiple load cases, showing the necessity to add kinematic constraints to the formulation, and to optimize a three dimensional strucure, showing the versatility of the formulation.

In \chpref{} we interest in how to implement modular constraints in the proposed \gls{tto} formulation. We decided to use a full-scale method--so a methodh that does not assume pysical scale separation between the module and the whole strucute-- called variable linking, in which the periodicity of the structure is created by linking the design variables of different subdomains of the structure. We improved this method by permitting to consider multiple modules' topology mathematically modeling this using the Kronecher product. In the following part of the chapter we put the formulation to the test by running a multitude of different numercial optimization in order to explore the limit and the trends of the modular tto optimization. First, we showed how closely related are multi moad cases and modular structures, underlying then the need for kinematic compatibility constraints for modular structurres. Then, we performed a parametric study on two hyperparameters of the modular optimization: the number of subdomains and the complexity of the module--\ie the number of candidates used to discretize a module. Using the numerical results of this test we constructed a Design of experiments (DOE), we provided some recommendations based on that: fewer subdomains are generally preferable, with the module as large as manufacturably possible. Module complexity plays a role in volume minimization but has a relatively low impact. Finally, the modular TTO structures are benchmarked against one of the most commonly used module topologies in the literature: the octet-truss lattice.

The modular results obtained till now showed an increased volume when compared to the correspective monolitich structure. This increase in mass is expected as we are repeating the same module over and over the design space, where there could be different loading consitions. the module is then obliged to show mechainical properties that works well everywhere in the structure. Trying to reduce the gap between modular and monolitic structures, in \chpref{} we added a new design variable to the problem: the layout of the modules in the space. The problem is then retrasformed in the following: we optimize the topology of multiple modules and concurrently we dispose then in the design domain. This problem is extremely more difficoult as there exist a strong connexion between the module topology and the layout, and due to the fact that the layout of the modules is inherently a discrete problem. The discrete design space is firstly relaxed using a continuous modelization and on which we used a modified \gls{dmo} approach, utilizing a gradient-based optimizer. To reduce the risk of converging towards non-pysical solutions-- such as solutions that presents a mix of modules topologies-- we developed a dual penalization scheme based on the RAMP method. Finding an appropriate starting point for the optimization proved to be non-trivial, as the topology of the modules is closely tied with the module layout, and vice versa. We employed the k-means clustering approch on the stress distribution of the unoptimized initial structure to provide a first module layput for the problem. THe proposed formulation is tested against multiple two and three dimensional test cases, finding better results than the literature. finally we showed how the control of the number of module topologies and the control on the presence or not of a subdomain is an effective ways to reduce the gab between monolitich and modular structures, all while maintaining the modular advntages. It is important however to remind that having more modules is tied with higher mnufacturing complexity. It will be then the user to decide the best fit value for the required application.

Up untill this poijnt, we presented only academic test cases. In \chpref{} we deal with two apllications taken from the aerospace domain. Initially, the monolithic optimization algorithm is used to reduce the weight of the wingbox of the \gls{crm}, a standard benchmark for aeronautic research. The test case is subjected to multiple load cases(+2.5g, -1g and cruise loads) associated with some correspective safety factors. The optimization is conducted using different materials and discretizations, resulting in lighter structures in less time compared to the literature. Later, the modular optimization formulation presented in \chpref{chap:06} is used on a drone-sized wing based on the 0012 NACA wing profile, showing how the proposed algorithm can be applied to complex real-world scenario test cses.

\section*{Perspectives}
The possibilities offered by this research are manifold. Some part concerns the methodology directly, with
research directions which were not explored, and others are about the future possibilities. we will review here a list of the possible perspectives grouped by chapter. 

In \chpref{} we only deal with generic density based topology optimization. It would have been interesting to see how the results would change if we compared to . We speculte, however, that as these methods still relies on the continuous discretization for the fea and for the sensitivity analysis, the results should not be too different from what we presented, expecially when comparing the computational time.

Considering the \chpref{}, we have dealed principally with local buckling constraints, but it would definetly be interesting ipmelemnt gobal buckling contraints as well. Then, we used always bar elements and not beam elements, negleting the influence of the joint stiffness in the optimization. We know that these assumptions hold for high value of slenderness, but it would be interesting to see what it happens if we make this swithc. Lastly, we added essentially multiple mechanical contratins to the optimization and the manufacturing
complexity is discussed here only as an outcome of the
optimization strategy. However, it would be important to consider additional manufacturing constarints (maximum numbers of bars converging to a single node,
minimum section, imposed periodicity of the structure) during the optimization would be beneficial. 

In \chpref{} we always considered cubic modules. However, the module external shape--as the ratio between the dimensions or the shape that could be pyramidal or dodecaedric-- has cenrtantly a big influence on the mechainical properties of the module itself. THis additional design variable of the modular problem should be investigated, balancing the tradeoff betweeen shape complexity, mass, and manufacturing complexity. It should be noted that thre exists multiple studies on the tassellation of 2d and 3d space, and these studies could be a good inspiration for working in this direction. Additionally, we have only considered the study of topological buckling inside the module, while it would be interesting study how to algorithmically implement topological buckling at the structure level.

\chpref{} ha messo in luce the big sfida in concurrently optimizing the modules' topology and layout. Even if the proposed pertubation of the starting point proved to work well in addressing this problem, a more complex heuristic could be beneficial for the optimization. One could for example solve an alternating formulation in wich first we optimize the layout and then the topology in a repetitive manner till convergence instead of the complex concuren optimization.


Finally, in \chpref{} we obseerved how important the choice of the initial ground strucutre is fo the optimization of real sized structures. Major studies should be conducted on how to concieve the best ground sructure for a specific problem.