\setchapterpreamble[u]{\margintoc}
\glsresetall % reset glossary

\chapter*{Conclusion and perspectives}
\phantomsection
\addcontentsline{toc}{chapter}{Conclusion and perspectives}
\markboth{Conclusion and perspectives}{Conclusion and perspectives}%

\section*{Conclusion}
In the aerospace sector, weight reduction plays a fundamental role due to the intricate relationship between weight and lift. The need to reduce aircraft weight is essential as it directly influences wing loading, thereby enhancing aerodynamic efficiency, maneuverability, and fuel efficiency. Beyond its economic significance, weight reduction has become a pressing environmental concern. In response to these challenges, innovative concepts such as the \gls{bwb} or the utilization of transonic truss-braced wings with high elongation have been proposed. These concepts share a common requirement for thin, lightweight wings with a high aspect ratio. A potential design solution fitting well within this context is the application of lattice structures as the primary structure for these wings. The inherent low mass and modular construction benefits make lattice structures an attractive choice. However, the challenge lies in the absence of a standardized method for their design and optimization for this kind of structure. Addressing these considerations, this thesis introduces the development of a design method and an optimization algorithm tailored for ultralight lattice structures.

In \chpref{chap:03}, we conducted a comprehensive comparison between two optimization frameworks for designing ultralight structures: density-based topology optimization and \gls{tto}. Initially, we formulated a shared volume minimization problem with material resistance constraints for both methods. Then, we explored the differences and similarities in their modeling, especially focusing on the disparities between the \gls{nand} and \gls{sand} approaches. We performed numerical tests using a two-dimensional L-shaped beam as a test case, where we gradually adjusted the material properties to achieve various volume fractions--\ie high strength is associated with low volume fraction. The results revealed two critical limitations in the applicability of these methods: density-based topology optimization faced challenges at low volume fractions due to the continuous discretization needing mesh refinement, leading to increased computational costs. Conversely, \gls{tto} encountered issues at moderately high volume fractions, where the cross-sectional area radius became too large, impacting the applicability of the truss idealization. Computational time observations highlighted \gls{tto}'s linear nature, providing clear advantages in this regard. Another important finding is that with continuous discretization, achieving finer details in the optimized structure demands more elements. In contrast, \gls{tto} with ground structure discretization eliminates this need, making the optimization process more straightforward and efficient. Considering these findings, we decided to focus our work on \gls{tto}, aligning well with our goal of optimizing ultralight structures.

In \chpref{chap:04}, we tackle the main limitations of the classical \gls{tto} method. Initially, we introduce a constraint on the minimum slenderness of active bars in optimized structures, expanding the method's range of applicability. We then focus on incorporating local buckling constraints, crucial for lightweight structures, additionally addressing the nodal stability of compressed bars, known as buckling chains or topological buckling. We then model kinematic compatibility constraints to handle more complex scenarios resulting in statically inadmissible structures, such as multiple load cases or imposed symmetries. We formulate the complete TTO approach with these additions, extendable to multiple load cases. However, naively implementing this on a \gls{nlp} solver proves challenging due to the problem's extreme multimodality. To overcome this, we proposed an innovative two-step optimization algorithm. The first step consists of the resolution of a relaxed problem to generate an initial approximate solution for subsequent complete optimization. The reinitialization heuristic reduces the starting point's influence on results. Through this approach, we demonstrate the algorithm's effectiveness on benchmark problems, showcasing its robustness and versatility in handling complex structures under various load cases. Remarkably, in specific case studies like the ten-bar truss, we demonstrated the algorithm's robustness, revealing negligible starting point influence through 100 random initializations. Additionally, when applied to the 2D cantilever beam problem proposed by Achtziger, the algorithm consistently found improved solutions. Furthermore, we extended the testing to more complex scenarios, including a two-dimensional truss subjected to multiple load cases. This highlighted the need to incorporate kinematic constraints into the formulation for accurate optimization. Finally, the algorithm showcased its versatility by successfully optimizing a three-dimensional structure, illustrating its adaptability to diverse engineering challenges.

\chpref{chap:05} is devoted to the development of the modular framework for the proposed \gls{tto} formulation. Opting for a full-scale method, a method that does not assume physical scale separation between the module and the whole structure, called variable linking, in which the periodicity of the structure is created by linking the design variables of different subdomains of the structure. Mathematically, we enable the consideration of multiple modules' topology by employing the Kronecker product. The subsequent section of the chapter involves a comprehensive examination of the formulation through various numerical optimizations to show the limits and trends of modular \gls{tto} optimization. First, we highlight the close relationship between multi-load cases and modular structures, emphasizing the need for kinematic compatibility constraints in modular structures. Then, a parametric study investigates two key hyperparameters of the optimization: the number of subdomains and the complexity of the module. Based on the numerical findings, a \gls{doe} is constructed, leading to recommendations such as favoring fewer subdomains with modules as large as manufacturably possible. Module complexity, while impacting volume minimization, exhibits a relatively low overall impact. To conclude, the modular \gls{tto} structures undergo benchmarking against the widely used octet-truss lattice.

Modular structures show a mass increase compared to their monolithic counterparts due to the repetition of the same module across the design space, where varying loading conditions may exist. To narrow the gap between modular and monolithic structures, in \chpref{chap:06} we introduce a new design variable: the layout of modules in space. The problem is reformulated to concurrently optimize the topology of multiple modules and their arrangement in the design domain. This task is significantly more challenging due to the strong connection between module topology and layout, compounded by the inherently discrete nature of the layout problem. The discrete design space is firstly relaxed by employing a continuous model and then applying a modified \gls{dmo} approach with a gradient-based optimizer. To prevent convergence towards non-physical solutions, a dual penalization scheme based on the \gls{ramp} method is developed. Finding an appropriate starting point is non-trivial, and we use the k-means clustering approach on the stress distribution of the unoptimized initial structure to provide an initial module layout. The proposed formulation is tested across various two and three-dimensional cases, demonstrating superior results compared to the literature. We show that controlling the number of module topologies and the presence of subdomains effectively reduces the gap between monolithic and modular structures while preserving modular advantages. However, it is essential to note that having more modules is associated with increased manufacturing complexity, and the user must decide the optimal value for their specific application.

In \chpref{chap:07}, we extend our analysis to real-world applications from the aerospace domain. Initially, we employ the monolithic optimization algorithm to reduce the weight of the wingbox of the \gls{crm}, a standard benchmark in aeronautic research. The test case is subjected to multiple load cases (+2.5g, -1g, and cruise loads) with respective safety factors. The optimization, conducted with various materials and discretizations, results in structures lighter and achieved in less time compared to existing literature. Subsequently, the modular optimization formulation introduced in \chpref{chap:06} is applied to a drone-sized wing based on the 0012 NACA wing profile. This showcases the versatility of the proposed algorithm in addressing complex real-world scenarios and test cases within the aerospace domain.

\section*{Perspectives}
The research opens up various possibilities and potential directions for future exploration. We can categorize these prospects based on the respective chapters.

In \chpref{chap:03}, we focused solely on generic density-based topology optimization. It would have been intriguing to compare the results with feature-mapping optimization methods. While we can only speculate, likely, the outcomes would not significantly differ since these methods also depend on continuous discretization for finite element analysis and sensitivity analysis as density-based methods. Any variations, especially in terms of computational time, may not be substantial.

Regarding \chpref{chap:04}, we primarily addressed local buckling constraints, but it would be valuable to explore the implementation of global buckling constraints. Additionally, we consistently used bar elements instead of beam elements, neglecting the impact of joint stiffness in the optimization. While our assumptions are valid for high slenderness values, it would be interesting to investigate the effects of this switch. Lastly, while multiple mechanical constraints were incorporated into the optimization, the discussion on manufacturing complexity is presented only as an outcome of the optimization strategy. It would be essential to consider additional manufacturing constraints, such as a maximum number of bars converging to a single node, or minimum section requirements of the structure during the optimization process. Additionally the material has always been used only as an input for the optimizatio. More studies neeed to be done to study how to include the material selection in the optimization loop as already done in \sidecite{duriez_properties_2022}.

In \chpref{chap:05}, we consistently used cubic modules; however, the external shape of the modules — \ie the ratio between dimensions or alternative shapes such as pyramidal or dodecahedral — could significantly influence the mechanical properties of the module. Investigating this additional design variable in the modular problem is essential, striking a balance between shape complexity, mass, and manufacturing complexity. It's worth noting that numerous studies on the tessellation of 2D and 3D space could serve as inspiration for exploring this direction \sidecite{coxeter_regular_1973, loeb_space-filling_1991}. Additionally, our focus on topological buckling was limited to within the module, and further exploration on algorithmically implementing topological buckling at the structure level would be intriguing.

\chpref{chap:06} has emphasized the significant challenge of concurrently optimizing the topology and layout of modules. Although the proposed perturbation of the starting point has proven effective in addressing this issue, a more complex heuristic could enhance the optimization process. For instance, an alternating formulation could be employed, where the layout and topology are optimized sequentially in a repetitive manner until convergence, offering an alternative approach to the intricacies of concurrent optimization.

Finally, in \chpref{chap:07}, it was observed that the choice of the initial ground structure significantly impacts the optimization of real-sized structures. Extensive studies should be conducted to determine the most effective approach for conceiving the optimal ground structure tailored to specific problems. Additionally, Till here we talked as the environmental cost only as an outcome of the optimization, but it is important to remember that they are very important to consider during the optimization. it could very interesting to add them in the optimization formulation to do eco design \sidecite{duriez_ecodesign_2022,duriez_co_2_2023,duriez_fast_2023}