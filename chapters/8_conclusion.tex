\setchapterpreamble[u]{\margintoc}
\glsresetall % reset glossary

\chapter*{Conclusion and perspectives}
\phantomsection
\addcontentsline{toc}{chapter}{Conclusion and perspectives}
\markboth{Conclusion and perspectives}{Conclusion and perspectives}%

\section*{Conclusion}
In the aerospace sector, weight reduction plays a fundamental role due to the intricate relationship between weight and lift. The need to reduce aircraft weight is essential as it directly influences wing loading, thereby enhancing aerodynamic efficiency, maneuverability, and fuel efficiency. Beyond its economic significance, weight reduction has become a pressing environmental concern. In response to these challenges, innovative concepts such as the \gls{bwb} or the utilization of transonic truss-braced wings with high elongation have been proposed. These concepts share a common requirement for thin, lightweight wings with \gls{har}. A potential design solution fitting well within this context is the application of modular lattice structures as the primary structure for these wings. The inherent low mass and modular construction benefits make these structures an attractive choice. However, the challenge lies in the absence of a standardized method for the design and optimization for this kind of structure. Addressing these considerations, this thesis introduces the development of a design and optimization methodology tailored for ultralight modular structures.

In \chpref{chap:03}, we conduct a comprehensive comparison between two optimization frameworks for designing ultralight structures: density-based topology optimization and \gls{tto}. Initially, we formulate a volume minimization problem with material resistance constraints for both methods. Then, we explore the differences and similarities in their modeling, especially focusing on the disparities between the \gls{nand} and \gls{sand} approaches. We perform numerical tests using a two-dimensional L-shaped beam as a test case, where we gradually adjust the material properties to achieve various volume fractions--\ie, high strength is associated with low volume fraction. The results reveal two critical limitations in the applicability of these methods: density-based topology optimization faces challenges at low volume fractions (<\qty{2}{\percent}) due to the need for very fine meshes, leading to increased computational costs. Conversely, \gls{tto} encounters issues at moderately high volume fractions (>\qty{7}{\percent}), where the cross-sectional areas of the structures become too large, departing from the truss idealization range of applicability. The computational time comparison strongly favors the \gls{tto} method, primarily because of the inherent linearity of its formulation. Another important point is that with continuous discretization, achieving finer details in the optimized structure demands more elements. In contrast, \gls{tto} with ground structure discretization eliminates this need, making the optimization process more straightforward and efficient. However, it is also observed that the initial ground structure could influence the optimized structure topology. Considering these observations, we decide to focus our work on \gls{tto}, aligning well with our goal of optimizing ultralight structures.

\chpref{chap:04} addresses the primary limitations of the classical \gls{tto} method by implementing multiple advanced mechanical constraints. Initially, a constraint on the minimum slenderness of active bars in optimized structures is introduced, expanding the method's range of applicability. We then focus on incorporating local buckling constraints, crucial for lightweight structures, additionally addressing the nodal stability of compressed bars, known as buckling chains or topological buckling. Kinematic compatibility constraints are modeled to handle more complex scenarios resulting in statically inadmissible structures, such as multiple load cases or imposed symmetries. A comprehensive TTO formulation is proposed by considering all these constraints, together with the extension to multiple load cases. However, naively solving the proposed formulation on a \gls{nlp} solver proves challenging due to the existence of multiple local optima. To overcome this, we propose an innovative two-step optimization algorithm. The first step involves resolving a relaxed problem to produce an initial approximate solution, which serves as the starting point for the second step optimization. In the second step, a complete formulation is applied, restoring the constraints that were relaxed in the first step. The reinitialization heuristic reduces the starting point's influence on results. We demonstrate the algorithm's effectiveness on benchmark problems, showcasing its robustness and versatility in handling complex structures under various load cases. Remarkably, in specific case studies like the ten-bar truss, we demonstrate the algorithm's robustness, revealing negligible starting point influence through 100 random initializations. Additionally, when applied to the 2D cantilever beam problem proposed by Achtziger, the algorithm consistently finds improved solutions, leading to a \qty{9}{\%} lower volume with respect to the best solution found in literature. Furthermore, we extend the testing to more complex scenarios, including a two-dimensional truss subjected to multiple load cases. This highlights the need to incorporate kinematic constraints into the formulation for accurate optimization. Finally, a three-dimensional structure is optimized, illustrating the proposed optimization method adaptability to diverse engineering challenges.

\chpref{chap:05} is devoted to the development of the modular framework for the proposed \gls{tto} formulation. We chose to employ a full-scale method, known as variable linking, which does not assume physical scale separation between the module and the entire structure. This approach creates the periodicity of the structure by linking the design variables of different subdomains within the structure. The successive section of the chapter focuses on testing the proposed optimization methodology to identify the limitations and trends of modular \gls{tto} optimization. First, we highlight the close relationship between multi-load cases and modular structures, emphasizing the need for kinematic compatibility constraints in modular structures. Then, a parametric study investigates two key hyperparameters of the optimization: the number of subdomains and the complexity of the modules. A \gls{doe} is constructed and utilized to illustrate the trends in modular optimization, leading to recommendations such as favoring fewer subdomains with modules as large as manufacturably possible. The complexity of modules, though it positively affects volume minimization by reducing volume as complexity increases, has a relatively lower overall impact compared to the number of subdomains.  In conclusion, we compare the performance of the modular TTO optimization method with a structure featuring the same number of subdomains but populated with the commonly used octet-truss lattice. The TTO optimized structure achieved a remarkable volume reduction ranging from over \qty{40}{\%} to \qty{60}{\%}, depending on the number of subdomains considered, compared to the octet-truss lattice under the same prescribed load case.

Modular structures show a mass increase compared to their monolithic counterparts due to the repetition of the same module across the design space, where varying loading conditions may exist. To narrow the gap between modular and monolithic structures, in \chpref{chap:06} we broaden the design space of the optimization by focussing on the spacial layout of the modules. The problem is reformulated to concurrently optimize the topology of multiple modules and their arrangement in the design domain. This task is significantly more challenging than the modular optimization with variable linking due to the strong connection between module topology and layout, compounded by the inherently discrete nature of the layout problem. A continuous optimization method is proposed to solve this problem. The discrete design space of the modules' layout optimization is firstly relaxed by employing a continuous model and then is optimized by applying an optimization method based on a modified \gls{dmo}. To prevent convergence towards non-physical solutions, a dual-phase penalization scheme based on the \gls{ramp} method is implemented. Finding an appropriate starting point is non-trivial, and we use the k-means clustering approach on the stress distribution of the initial structure to provide a starting point for the module layout. The proposed formulation is tested across various two and three-dimensional cases, demonstrating superior results compared to the literature. We show that controlling the number of module topologies and the layout of modules effectively reduces the gap between monolithic and modular structures. Remarkably, on the simply supported 3D beam we achieved a close volume (+\qty{2.8}{\percent}) to the monolithic reference while preserving the manufacturing advantages inherent to the modular nature. However, it is essential to note that having more modules is associated with increased manufacturing complexity, and the user must decide the optimal value for their specific application by using a multicriteria analysis.

In \chpref{chap:07}, we extend our analysis to real-world applications from the aerospace domain. Initially, we employ the monolithic optimization algorithm to reduce the weight of the wingbox of the \gls{crm}. The test case is subjected to three load cases with an associated safety factor. The optimization, conducted with ground structures, results in structures that are lighter (-\qty{27}{\%}) and optimized in less time compared to existing literature (minutes instead of days). Moreover, we performed additional tests, such as quantitative studies on the influence of the material, the addition of maximum displacement constraints, and the influence of the initial ground structure on the optimized solution. Subsequently, the modular optimization formulation introduced in \chpref{chap:06} is applied to a drone-sized wing based on the 0012 NACA wing profile. This showcases the versatility of the proposed algorithm in addressing complex real-world scenarios and test cases within the aerospace domain.

\section*{Perspectives}
The research opens up various possibilities and potential directions for future exploration, which are categorized according to the corresponding chapters.

In \chpref{chap:03}, we focused on the comparison between TTO and solely the generic density-based topology optimization. It would have been intriguing to compare the results with feature-mapping optimization methods. While we can only speculate, likely, the outcomes would not significantly differ since these methods also depend on continuous discretization for finite element analysis and sensitivity analysis as density-based methods. Any variations, especially in terms of computational time, may not be substantial.

Regarding \chpref{chap:04}, we primarily addressed local buckling constraints, but it would be valuable to explore the implementation of global buckling constraints. Additionally, we consistently used bar elements instead of beam elements, neglecting the impact of joint stiffness in the optimization. While our assumptions are valid for high slenderness values of bars, it would be interesting to investigate the effects of this switch. Then, while multiple mechanical constraints were incorporated into the optimization, the discussion on manufacturing complexity is presented only as an outcome of the optimization strategy. It would be interesting to consider additional manufacturing constraints, such as a maximum number of bars converging to a single node, or minimum section requirements of the structure during the optimization process. lastly, materials have thus far been treated solely as inputs for optimization. Further research is required to explore methods for integrating material selection into the optimization loop, as demonstrated in prior work \sidecite{duriez_properties_2022}.

In \chpref{chap:05}, we consistently used cubic modules; however, the external shape of the modules—\ie the ratio between dimensions or alternative shapes such as pyramidal or dodecahedral—could significantly influence the mechanical properties of the module. Investigating this additional design variable in the modular problem is important, striking a balance between shape complexity, mass, and manufacturing complexity. It's worth noting that numerous studies on the tessellation of 2D and 3D space could serve as inspiration for exploring this direction \sidecite{coxeter_regular_1973, loeb_space-filling_1991}. Additionally, our focus on topological buckling was limited to within the module, and further exploration on algorithmically implementing topological buckling at the structure level would be intriguing.

\chpref{chap:06} has emphasized the significant challenge of concurrently optimizing the topology and layout of modules. Although the proposed perturbation of the starting point has proven effective in addressing this issue, a more complex heuristic could enhance the optimization process. For instance, an alternating formulation could be employed, where the layout and topology are optimized sequentially in a repetitive manner until convergence, offering an alternative approach to the intricacies of concurrent optimization.

Finally, in \chpref{chap:07}, it was observed that the choice of the initial ground structure significantly impacts the optimization of real-sized structures. Extensive studies should be conducted to determine the most effective approach for conceiving the optimal ground structure tailored to specific problems. Additionally, up to this point, we have primarily discussed environmental costs as an outcome of optimization. However, it is crucial to consider them during the optimization process. Incorporating environmental costs into the optimization formulation could offer compelling opportunities for eco-design initiatives \sidecite{duriez_ecodesign_2022,duriez_co_2_2023,duriez_fast_2023}.