\setchapterpreamble[u]{\margintoc}
\glsresetall % reset glossary

\chapter{Literature review}
Introduction
\section{An introduction to structural optimization}

\section{Ultra-lightweight structures optimization}

\subsection{Feature-Mapping topology optimization}

\subsection{Truss topology optimization}

\section{Cellular structures optimization}
why modular, which are the advantages

define modules and subdomains

intro there are different type of approach that we could use, full scale and multiscale

intro multiscale.

numerical homogenization

intro full scale

the problem on the number of subsections needed to correctly foresee the mechanical behaviour

\subsection{Multi-scale structures optimization}

\subsection{Full-scale structures optimization}

Although the aforementioned methods are able to design a structure consisting of a simple module repeated several times in the global domain, they suffer from a common limitation. Namely, the designs converge towards solutions with compromised structural performance (Huang and Xie 2008; Zhang and Sun 2006). The cause lies within the topological periodicity. The topology of the module is influenced most by the region with the highest compliance. The resulting module design is used at different locations in the structure, therefore not leading to the most optimal solution for these regions (Tugilimana et al. 2019). This shortcoming can be addressed by two approaches: (i) by defining additional module properties as design variables or (ii) by allowing more modules within the structure. Both approaches extend the solution space. The first approach, extending the solution space, was considered by allowing for rotation of a module. Allowing for rotations resulted in improved structural performance because rotation of the modules modifies the material distribution in the structure locally (Tugilimana et al. 2017). Also, in a 2D continuum setting, the one-to-one mapping of a module unit to the global domain is extended by by allowing the module unit to resize (Stromberg et al. 2011).  \sidecite{bakker_simultaneous_2021}