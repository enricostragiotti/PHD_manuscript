\section*{Abstract}

\chapter*{Introduction}
\addcontentsline{toc}{chapter}{Introduction}

\textit{Scientists study the world as it is,}\\
\textit{Engineers create the world that never has been.} \vspace{5pt} \\
--- Theodore von K\'arm\'an \\

\section*{Towards lighter structures}

In the aerospace industry, an ongoing demand exists for lighter aerostructures, motivated by the imperative to enhance fuel efficiency and overall performance. This emphasis on lighter structures and materials not only reduces operational costs for airlines but also aligns with a broader commitment to sustainability, mitigating fuel consumption and carbon emissions. Furthermore, the aerospace sector is currently witnessing two innovative shifts: the transition to hydrogen-powered and electric planes, directing engineering efforts toward cleaner and more sustainable aviation technologies. These changes offer opportunities to deviate from the classic tube-and-wing configuration and explore inventive concepts like the flying wing or transonic truss-braced wings. Regardless of the specific configuration, a highly probable shared goal is the necessity to redesign lightweight dry wings with high aspect ratios and thin profiles.

One promising approach to meet these requirements is the utilization of lattice structures. These structures not only offer ultralight properties but also modularity. Modular designs bring numerous advantages, including the ability to assemble large structures from smaller and easier-to-manufacture repeating modules, on-field reparability, and rapid assembly for temporary structures.



By addressing the challenges of lightweight design and modularity in aerostructures, this research aims to contribute to the ongoing evolution of aerospace technologies and advance the efficiency and performance of future aircraft.

Lattice structures, though potent, present practical considerations for engineers and designers in real-world applications. In manufacturing, additive techniques prove more efficient for intricate non-planar lattice structures, urging careful alignment with economic factors and material choices unique to additive manufacturing. Simulation challenges arise in stress simulations for expansive lattice structures, potentially requiring physical testing for precise evaluations.

File size issues surface during the conversion of designs with substantial lattice sections to STL files, hindering processing for all but the most powerful computers. While reducing mesh size is an option, caution is needed to avoid oversimplification and visible triangular elements on the final product.

Furthermore, the limited array of readily available cell types restricts the versatility of lattice structures. While certain software allows new cell type creation, this remains a highly specialized and technical endeavor.

In summary, realizing the potential of lattice structures in real-world applications demands a nuanced understanding of manufacturing methods, simulation challenges, file size implications, and limitations in available cell types. Engineers must navigate these considerations to harness the benefits of lattice structures effectively.



\section*{Objective}
The primary objective of this thesis is to develop a design and optimization algorithm for ultralight and modular aerostructures. 

\section*{Outline of the thesis}
In the initial phase, we reviewed existing literature to identify the most suitable basis for optimizing monolithic (non-modular) structures. After a thorough comparison, we selected the Truss Topology Optimization (TTO) approach, an optimization method based on the use of bars as the discretizing element of the structure. However, the classic TTO formulation has limitations, such as the inability to address buckling constraints, consider multiple load cases, and ensure mechanical compatibility. To overcome these challenges, we proposed an innovative two-step optimization algorithm. In this approach, a relaxed problem is utilized to generate an initial solution, serving as the starting point for the optimization using a complete formulation.

The second part of the thesis focuses on adapting the proposed monolithic formulation to incorporate modular constraints. Initially, the emphasis is on optimizing the topology of a fully-modular structure, where a single module is repeated throughout the entire design. We evaluate how hyperparameters, such as the number of subdomains and module complexity, affect the mechanical performance of the structure. Subsequently, we delve into a more complex scenario, optimizing multiple module topologies and their layout within the structure. This is achieved through a Discrete Material Optimization (DMO) approach, employing a gradient-based optimizer.