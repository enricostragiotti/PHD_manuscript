\section*{Abstract}

\chapter*{Introduction}
\addcontentsline{toc}{chapter}{Introduction}

\textit{Scientists study the world as it is,}\\
\textit{Engineers create the world that never has been.} \vspace{5pt} \\
--- Theodore von K\'arm\'an \\

\section*{Towards lighter structures}



Lattice structures, though potent, present practical considerations for engineers and designers in real-world applications. In manufacturing, additive techniques prove more efficient for intricate non-planar lattice structures, urging careful alignment with economic factors and material choices unique to additive manufacturing. Simulation challenges arise in stress simulations for expansive lattice structures, potentially requiring physical testing for precise evaluations.

File size issues surface during the conversion of designs with substantial lattice sections to STL files, hindering processing for all but the most powerful computers. While reducing mesh size is an option, caution is needed to avoid oversimplification and visible triangular elements on the final product.

Furthermore, the limited array of readily available cell types restricts the versatility of lattice structures. While certain software allows new cell type creation, this remains a highly specialized and technical endeavor.

In summary, realizing the potential of lattice structures in real-world applications demands a nuanced understanding of manufacturing methods, simulation challenges, file size implications, and limitations in available cell types. Engineers must navigate these considerations to harness the benefits of lattice structures effectively.



\section*{Objective}

\section*{Outline of the thesis}